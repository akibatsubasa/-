\documentclass[12pt]{jreport}
\renewcommand{\bibname}{参考文献} 
\usepackage[dvipdfmx]{graphicx}
\usepackage{ascmac,url,moreverb,multirow,style/eclbkbox,fancybox,enumerate,bxbase,inputenc,listings}
\setlength{\textheight}{25cm}		%1ページ当りの行数を指定する
\setlength{\textwidth}{38zw}		%1行あたりの文字数の設定
\setlength{\evensidemargin}{10mm}   %偶数ページの余白
\setlength{\oddsidemargin}{10mm}    %奇数ページの余白
\setlength{\topmargin}{-3mm}        %上の余白
\setlength{\headheight}{0mm}        %ヘッダ領域の高さ
\setlength{\headsep}{0mm}           %ヘッダ領域と本文領域との間隔
\setlength{\columnsep}{12mm}        %段組にした場合の段同士の間隔
\setlength{\footskip}{10mm}         %フッタ領域と本文との間隔
\setlength{\belowcaptionskip}{5pt}
\input{macro.tex}					%マクロの読み込み
%%%%%%%%%%%%%%%%%%%%%%%%%%%%%%%%%%%%%%%%%%%%%%%%%%%%%%%%%%%%%%%%%%%%%%%

\title{AIを用いた楽曲制作に関する検討}					%卒業論文タイトル
\author{1532117 秋場  翼\\1532151 松元 孝樹\\\normalsize 指導教員:中村 直人 教授}	%名前(苗字と名前は全角1字空け)
\date{平成31年度}                   %日付設定(デフォルトは現在の年月日)

%%%%%%%%%%%%%%%%%%%%%%%%%%%%%%%%%%%%%%%%%%%%%%%%%%%%%%%%%%%%%%%%%%%%%%%
\begin{document}
\pagenumbering{roman}
\maketitle                        	%タイトルをドキュメントへ貼り付け
\tableofcontents               	%目次を作成
\listoffigures				%図の目次作成
\listoftables				%表の目次作成

\baselineskip 20pt              	%行間設定

\clearpage
\pagenumbering{arabic}


%ここから内容

%\input{ディレクトリ名/ファイル名}
\chapter{序論}
\section{研究の背景と目的}
近年,AI分野は急速な発展を続けている.
%第1次AIブームについて述べる
第一次ブームはコンピュータができ始めた1950~60年代です。コンピュータの登場により人間を超えるようなAIが誕生すると期待され、次々と新しいアルゴリズムが考案されました。第一次ブームの特徴として「推論と探索」があります。「推論と探索」とはコンピュータがゲームやパズルを解いたり、迷路のゴールへの生き方を調べるなどの技術のことです。第一次ブームの研究により生み出された新しいアルゴリズムにより、一見知的な活動を行えるようになりましたが、コンピュータの性能は低く、ルールとゴールが厳密に決まっている枠組のなかでしか動けないため、現実世界では全く役に立たないことが見えてきました。その結果、第一次AIブームは終結してしまいます。これらの第一次ブームでの人工知能のことをトイプロブレム(おもちゃの問題)と呼ばれます。コンピュータの性能の限界が見えたことから、1970年代に一回目の冬の時代に突入していきます.\\
%第2次AIブームについて述べる
1980年代に入り、家庭にコンピュータが普及したことにより第二次ブームが発生しました。第二次AIブームの特徴として「エキスパートシステム」が挙げられます。「エキスパートシステム」とは専門家の知識をコンピュータに教え込みことで現実の複雑な問題を人工知能に解かせることを試みたシステムです。第一次ブームと比較してコンピューターの小型化・性能が高まっており、ある程度はこれらの試みは成功しましたが、知識を教え込む作業が非常に煩雑であること、例外処理や矛盾したルールに柔軟に対応することが出来ませんでした。日常世界を見渡してみると、これらの例外処理や矛盾したルールは非常に多く、知識を教え込む作業が非常に困難なことから、第二次AIブームは自然に消滅へと向かってしまいました。その後、1990年代半ばにWindows95の登場、インターネットの普及、検索エンジンの高性能化が進み、世界中にいる誰もが簡単に大量のデータを扱える時代に突入しました.\\
%第3次AIブームについて述べる
第二次AIブームでのエキスパートシステムが壁にぶつかった問題として、日常世界には例外処理や矛盾したルールが非常に多く、知識を教え込む作業が非常に困難というのがありました。これはコンピュータはプログラムと呼ばれるあらかじめINPUTされた命令を順次行っていくため、INPUTされていない例外処理や、矛盾したルールにぶち当たった場合に柔軟に対応出来なかったためです。これらを解決する手段として「機械学習」や「ディープラーニング」にてコンピュータが自らが学んでいくという手法が第二次AIブームの時代から研究されていましたが、実用化するためにはコンピュータの性能が追い付いていませんでした。しかし、2000年代に入り、コンピューターの小型化・性能向上に加えインターネットの普及、クラウドでの膨大なデータ管理が容易となったことで実現可能なレベルとなり、第三次AIブームが沸き起こりました。第三次AIブームでの主な出来事1997年:チェス専用のコンピューターが世界王者に勝利2006年:ディープラーニングの実用方法が登場2011年:IBMワトソンがクイズ番組で人間に勝利する2012年:画像認識の向上で画像データから「猫」を特定できるようになる2016年:「アルファ碁」がプロ棋士に勝利を収める2016年はディープラーニングを起爆剤としたAIが社会に衝撃を与え急速に発達した年であると言われています。次の年の2017年が実用的なシステムも世の中に登場し始めてきており「AI元年」と呼ばれています。このようにAIブームは1950年代の第一次AIブーム、1980年代の第二次AIブームと盛り上がっては衰退していきましたが、数々の実用的なシステムの登場により第三次AIブームは継続して続いていくだろうと予想されています.\\
\begin{figure}[h]
    \begin{screen}
    \begin{center}
        \includegraphics[scale=0.4, clip]{./img/AI_History.png}
        \caption{第一次AIブームから今までの流れ\newline(引用:https://bit.ly/2Wc9Ykb)}
        \label{fig:第一次AIブームから今までの流れ}
    \end{center}
\end{screen}
\end{figure}\\
スマートスピーカなどの対話型のAIがGoogleやAmazon,IBMによって商品化され,現在ではスマートフォンにもSiriというAIが搭載されるなどその存在は非常に身近になっており,その種類も非常に多岐にわたる.
\begin{figure}[h]
    \begin{screen}
    \begin{center}
        \includegraphics[scale=0.6, clip]{./img/smartspeaker_list.jpg}
        \caption{多種多様なスマートスピーカー}
        \label{fig:多種多様なスマートスピーカー}
    \end{center}
\end{screen}
\end{figure}\\
また囲碁や将棋,チェスなどの競技においても,プロにAIが勝利するなどその精度は以前から高いが,そのAIは一つの競技でしか使用できない特化型人工知能(AGI)でありった.
しかし,英DeepMindが発表したAlphaZeroという様々なボードゲームに対応できる汎用性を持ったAIが発表され,汎用人工知能(GAI)の成長も著しい.\\
自然言語処理を用いた芸術の分野では,2012年にスタートした人工知能を使って小説を生成するプロジェクトが「星新一賞」の第一審査を通過した.
また,絵画や音楽に関してもAIが作成した肖像画が米競売大手クリスティーズのオークションで43万2500ドル(約4900万円)で落札され,AIを用いて新しい作品を作るものが出回っている.\\
 このようにAIの発展は様々な分野においてその成果を上げており,今後は業務の効率化や補助だけにとどまらず,自動車の自動運転や医療の現場でも人間の手よりも高精度なものとして活躍することが期待されている.\\
本研究ではAIによる楽曲生成についての実証実験を行う.
Googole brainによって公開されているTensorflowのライブラリであるMagentaはAI Duetや
そのライブラリを用いて学習データやノード数による楽曲の生成結果の違いを比較,検証し,AIによる楽曲制作が有用なものか調査する.\\
\section{本論文の構成}
本論文の構成は以下の通りである.\\
第1章では本論文の背景と目的について述べている.\\
第2章では本論文で利用する理論について述べている.\\
第3章では実験内容について述べている.\\
第4章では楽曲制作について述べている.\\
第5章ではAIを用いた楽曲制作についての本研究の結論について述べている.\\
\chapter{理論}
\section{AIの理論}
%下2章の(AIの理論)にする
\subsection{AIの始まり}
AIの概念の始まりは,1950年にアラン・チューリングが提唱したチューリングテストである.\cite{ronbun1}
チューリングテストとは,男性,女性,質問者の3人でおこなわれ,目的は質問者がどちらが男性でどちらが女性かを当てるゲームである.会話は文字だけでおこない,男性は質問者を間違わせるように振舞い,逆に女性は質問者を助けるように振舞う.
次に,このゲームを男性の代わりに機械がおこない,質問者が二人の人間を相手にした時と同じくらいの間違いを人間と機械のペアにも起こすことができれば,機械は知性を持っているとするものである.
また,チューリングテストを受ける機械はどんな技術を使っても良く,機械を作った人がその機械の動作の仕様をきちんと説明できないようなものであってもかまわないということを認めている.
この概念が提唱されてから初めてチューリングテストに合格したのは,2014年にレディング大学で開催された「Turing Test 2014」で発表された,ウクライナ在住の13歳の少年が開発した「Eugene Goostman」というプログラムだった.
\begin{figure}[!ht]
    \begin{screen}
    \begin{center}
        \includegraphics[scale=0.6, clip]{./img/Eugene_Goostman.jpg}
        \caption{チューリングテストを合格したEugeneGoostman\newline(引用:http://www.itmedia.co.jp/news/articles/1807/26/news014.html)}
        \label{fig:チューリングテストを合格したEugeneGoostman}
    \end{center}
\end{screen}
\end{figure}
\subsection{第1次AIブーム}
その後,次のAIブームは1950年代後半から1960年代に起きた.
この時代になり初めてAI(Artificial Intelligence,人工知能)という言葉がダートマス会議で用いられた.\\
この会議は1956年に米ニューハンプシャー州のダートマス大学で開催され,コンピュータ研究者たちの研究成果を発表し合う研究発表会である.\cite{webpage4}
この会議の発起人であるジョン・マッカーシー氏がAIという言葉をThe Dartmouth Summer Research Project on Artificial Intelligence\cite{ronbun2}の中で用いた.
また,この会議で初めての人工知能プログラムと言われる“Logic Theorist”と呼ばれる数学原理をコンピュータで証明するデモンストレーションが行われた.
当時のコンピュータはせいぜい四則演算が限界だったので,これは画期的な成果といえた.\\
 他にも,マサチューセッツ工科大学のジョセフ・ワイゼンバウム氏が1966年に作成した単純な自然言語処理プログラム「ELIZA」が発表された.
ただ,ELIZAはそのプログラムはパターン照合を適用しているので,パターンにない会話や曖昧な事柄に対応できない.\\
 また,人工知能の理解は文字列だけにしか及ばず,画像の特徴を自己で判断し抽出することができないシンボルグラウンティング問題が指摘される.
例えば鳩とツノドリの画像を見せ分類するとする時,人間はツノドリのクチバシや足の色などの特徴から分類する.
人工知能はことの時,ツノメドリの特徴として何に着目したらいいのか分からないということが起きるため分類に失敗するという問題がおきた.
\subsection{第2次AIブーム}
1980年代に入り,家庭にコンピュータが普及したことにより第二次ブームが発生した.
「知識」(コンピューターが推論するために必要な様々な情報を,コンピューターが認識できる形で記述したもの)を与えることで人工知能が実用可能な水準に達し,
多数のエキスパートシステムとよばれる,専門分野の知識を取り込んだ上で推論することで,その分野の専門家のように振る舞うプログラムが生み出された.
日本では,政府による「第五世代コンピュータ」と名付けられた大型プロジェクトが推進された.
しかし,当時はコンピューターが必要な情報を自ら収集して蓄積することはできなかったため,必要となる全ての情報について,人がコンピューターにとって理解可能なように内容を記述する必要があり,
世にある膨大な情報全てを,コンピューターが理解できるように記述して用意することは困難なため,実際に活用可能な知識量は特定の領域の情報などに限定する必要があった.
こうした限界から,1995年頃からブームは衰えた.
\subsection{第3次AIブームの始まり}
第二次AIブームでのエキスパートシステムが壁にぶつかった問題として,日常世界には例外処理や矛盾したルールが非常に多く,知識を教え込む作業が非常に困難というのがあった.
これらを解決する手段として「機械学習」や「ディープラーニング」にてコンピュータが自らが学んでいくという手法が第二次AIブームの時代から研究されていたが,実用化するためにはコンピュータの性能が追い付いていなかった.\\
 しかし,2000年代に入り,コンピューターの小型化・性能向上に加えインターネットの普及,クラウドでの膨大なデータ管理が容易となったことで実現可能なレベルとなり,
2006年にはニューラルネットワークの代表的な研究者であるジェフリー・ヒントンらの研究チームが,制限ボルツマンマシンによるオートエンコーダの深層化に成功し,再び注目を集めた.
この際に発表した論文から,これまでの多層ニューラルネットよりもさらに深いネットワーク構造を意味するディープネットワークの用語が定着し,第三次AIブームが沸き起こった.
他にも,2011年にIBMのワトソンが難解な質問と独特の解答方法で知られる人気クイズ番組「ジョパディ!」に出演した.\cite{webpage6}
出演したワトソンが読み込んだ本や映画の脚本,百科事典などは合計100万冊にものぼり,公平を期すため,インターネットには接続しておらず,読み込んだデータのみでの勝負となり,歴代チャンピオン2人に勝利した.
\begin{figure}[!ht]
    \begin{screen}
    \begin{center}
        \includegraphics[scale=1.1, clip]{./img/Watson.jpg}
        \caption{クイズ番組に出演するWatson\newline(http://www.swift-web.org/cp-bin/blogn/index.php?e=1023)}
        \label{fig:クイズ番組に出演するWatson}
    \end{center}
\end{screen}
\end{figure}\\
2012年にはGoogleが発表した機械学習の論文では,事前に猫をネットワークに教えたわけでもなく,猫のラベル付けした画像を与えたわけではない人工知能を発表した.\cite{ronbun3}
このときまで,機械学習のほとんどは,ラベル付きデータの量に依存していた.
この論文により,機械がラベルのない生データでも処理することができ,そしておそらく,人間が予備知識を持たないデータですら処理できることが示された.
\subsection{第3次AIブームの現在}
2014年にはAmazon.comからスマートスピーカと呼ばれる対話型の音声操作に対応したAIアシスタント機能を持つスピーカーであるAmazon Echoが発売され,その後GoogleやAmazon,IBMによって様々なスマートスピーカーが商品化された.
現在ではスマートフォンにもSiriというAIが搭載されるなどその存在は非常に身近になっており,その種類も非常に多岐にわたる.
\begin{figure}[!ht]
    \begin{screen}
    \begin{center}
        \includegraphics[scale=0.6, clip]{./img/smartspeaker_list.jpg}
        \caption{多種多様なスマートスピーカー}
        \label{fig:多種多様なスマートスピーカー}
    \end{center}
\end{screen}
\end{figure}\\
\newpage
囲碁や将棋,チェスなどの競技においても,プロにAIが勝利するなどその精度は以前より高いが,そのAIは一つの競技でしか使用できない特化型人工知能(AGI)であった.
しかし,英DeepMindが発表したAlphaZeroという様々なボードゲームに対応できる汎用性を持ったAIが発表され,以降,汎用人工知能(GAI)の成長も著しい.\\
 自然言語処理を用いた芸術の分野では,2012年にスタートした人工知能を使って小説を生成するプロジェクトが「星新一賞」の第一審査を通過した.\cite{webpage2}
また,芸術の分野に関してもAIが作成した肖像画が米競売大手クリスティーズのオークションで43万2500ドル(約4900万円)で落札されたり,AIを用いて新しい楽曲を作るものが出回っていたりと,成長が著しい.\cite{webpage3}\\
\newpage
\section{AIを用いた楽曲作成}
作曲の流れはその構成によって階層化されており,比較的自動化が容易とされている.音楽を構成する3要素はスケール(調)とコード(和音)とメロディ(旋律)とされており,スケールが決まればその構成音に合わせてコードも決まる.コード進行は一定のルールがあり,これまでの曲の中で良い進行とされるパターンは数多く蓄積されている.コードは曲のムードを大きく左右し、
人間のその時の感情、感じ方に非常に影響を与えると言われている.メロディも同様にコードの構成音を元にすれば大きく外れることはないが,単調になってしまう傾向があるので,多少のランダムさが必要とされている要素である.コードに対して大きく外さない範囲で変動させれば単調になるのを防ぐこともできる.
%MIDIである必要性:なぜこれを使うのか波形でない理由
\subsection{MIDI}
音楽データには大きく分けて二つある.一つ目はオーディオデータといい波形の情報を記録する形式であり,二つ目がMIDIである.\\
 AIによる曲制作では主にMIDIファイルの音楽データを使用する.MIDIファイルは実際の音ではなく音楽の演奏情報(音の高さや長さなど)である.
本研究で用いるAIはこのMIDIファイルの情報を元に学習をする.また入出力の際もこの規格を用いる.\\
 なお,インプットデータはone-hot Vectorで図\ref{fig:インプットデータの仕組み}のようになっている.また,楽曲制作の際に音程を指定する場合は図\ref{fig:MIDIと音階}の数値を指定する.
\begin{figure}[h]
    \begin{screen}
    \begin{center}
        \includegraphics[scale=0.8,clip]{./img/midi1.png}
        \caption{インプットデータの仕組み}
        \label{fig:インプットデータの仕組み}
    \end{center}
    \end{screen}
\end{figure}
\newpage
\begin{figure}[h]
    \begin{screen}
    \begin{center}
        \includegraphics[scale=0.4,clip]{./img/midi2.png}
        \caption{MIDIと音階}
        \label{fig:MIDIと音階}
    \end{center}
    \end{screen}
\end{figure}
%AIを用いた楽曲サービスについて述べてその中でなぜMagendaを選ぶのかを述べる
\newpage
\subsection{Magenta}
本研究で使用するMagentaは音楽などをTensorFlowを使って機械学習するライブラリであり,Google BrainがGitHub上に公開されているOSSである.これを図\ref{fig:GitHub上に公開されているMagenta}に示す.\\
 Magentaではまず学習させたい音楽のMIDIデータをNoteSequence(magentaが扱うファイル形式)とよばれるデータフォーマットに変更する.それを学習用データセットと評価用データセットに変換したあと学習を行う.
このとき,一度に学習させるデータの数,学習を行う回数,ノード数を設定する.これをパッケージ化し,MIDIファイルとして新たに楽曲を生成するという流れである.これを図\ref{fig:magentaによるMIDI音楽データ生成までのプロセス}に示す.
\begin{figure}[h]
    \begin{screen}
    \begin{center}
        \includegraphics[scale=0.3, clip]{./img/magentagithub.png}
        \caption{GitHub上に公開されているMagenta}
        \label{fig:GitHub上に公開されているMagenta}
    \end{center}
    \end{screen}
\end{figure}
\newpage
\begin{figure}[h]
    \begin{screen}
    \begin{center}
        \includegraphics[scale=1.7, clip]{./img/magenta_usestep.png}
        \caption{magentaによるMIDI音楽データ生成までのプロセス}
        \label{fig:magentaによるMIDI音楽データ生成までのプロセス}
    \end{center}
    \end{screen}
\end{figure}
\subsection{リカレントニューラルネットワーク(RNN)}
本研究で用いるMagentaのモデルはRNN(Recurrent Neural Network)を取り入れたものである.
RNNは、ある層の出力がもう一度その層へ入力される回帰結合を持つニューラルネットワークである.
このような結合を持つことで、ニューラルネットは過去の情報を保持することができるようになる.
\newpage
\section{機械学習に適した開発環境について}
Tensolflowのランタイムとして以下のシステムがサポートされている.
\begin{enumerate}
    \renewcommand{\labelenumi}{(\arabic{enumi})}
    \item Ubuntu 16.04以降
    \item macOS 10.12.6(Sierra)以降(GPUサポートなし)
    \item Windows7 以降
    \item Raspbian 9.0以降
\end{enumerate}
 また,GPUを用いて学習を行う時にはtensolflow-gpuというパッケージが必要となり,導入には以下のドライバやライブラリが必要である.
\begin{enumerate}
    \renewcommand{\labelenumi}{(\arabic{enumi})}
    \item CUDA Toolkit (tensolflowはCUDA9.0をサポート)
    \item CUPTI (CUDA Toolkitに付随)
    \item NVIDEAGPUドライバ(CUDA9.0には384.x以上が必要)
    \item cuDNN SDK
\end{enumerate}
 Windows10の環境ではリリースされているCUDAのバージョンは10のみであり,Tensolflowのサポートを外れてしまう.
そのため,CUDAv9がインストール可能なUbuntuを用いることとし,システムの開発環境を表\ref{tab:開発環境}に示す.
\begin{table}[h]
\begin{center}
\caption{開発環境}
\label{tab:開発環境}
\begin{tabular}{|c|p{20zw}|}
\hline
    OS & Ubuntu 16.04 LTS\\
    \hline
    CPU & Intel Core i3 8100\\
    \hline
    メモリ & 8GB\\
    \hline
    GPU & GeForce GTX 1060\\
    \hline
    CUDA & CUDA(9),cuDNN(7.4.2)\\
    \hline
    ライブラリ & TensorFlow(1.12.0),magenta(0.5.0)\\
    \hline
\end{tabular}
\end{center}
\end{table}\\
\newpage
\subsection{CUDA}
CUDAは,NVIDIAが開発しているGPU上でプログラミングをするためのソフトウェアプラットフォームである.
含まれるものとしては,CUDAを実行形式に変えるコンパイラや,それをサポートするSDK,ライブラリ,デバッグツール群である.
CUDAを導入することによって,プログラムを複数のプロセッサで動かすだけでなく,無駄なく並列化することができる.
\chapter{実験内容}
\section{モデルによる違い}
 本研究ではMagentaで用意されている2つの学習モデルを用いた。\\
 MelodeRNNは楽曲のメロディを制作するモデルである.
\section{学習回数による違い}
\section{ノード数による違い}
\chapter{楽曲制作}
\section{Melody\_rnnを使用して学習モデルを作成}
\subsection{NoteSequenceの作成}
NoteSequenceとはMIDIデータから作成されるプロトコルバッファである.
プロトコルバッファとはGoogleが2008年にオープンソース化したバイナリベースのデータフォーマットである.
既存の技術としてはXMLやJSONなどのテキストベースのデータフォーマットがあるが,プロトコルバッファはバイナリフォーマットであるので,アプリケーション間でデータ構造の送受信をする際に少ないデータ量ですむという特徴がある.
\begin{table}[h]
\begin{center}
\caption{JSONとプロトコルバッファの比較}
\label{tab:JSONとプロトコルバッファの比較}
\begin{tabular}{|l|c|c|}
\hline
    & プロトコルバッファ & JSON\\ \hline
    \hline
    フォーマット & バイナリベース & テキストベース\\
    \hline
    データ量 & 少ない & 多い \\
    \hline
     &GPU & GeForce GTX 1060\\
    \hline
     &CUDA & CUDA(9),cuDNN(7.4.2)\\
    \hline
     &ライブラリ & TensorFlow(1.12.0),magenta(0.5.0)\\
    \hline
\end{tabular}
\end{center}
\end{table}\\
 NoteSequenceの作成は以下に示すコマンドで作成できる.\\
 --input\_dirで学習させるMIDIデータのディレクトリの絶対パスを指定し,--output\_fileでNotesequenceの出力先のディレクトリを指定する.\\
\begin{lstlisting}[basicstyle=\ttfamily\footnotesize,frame=single]
    convert\_dir\_to\_note\_sequences \
    --input\_dir=\$INPUT\_DIRECTORY \
    --output\_file=\$SEQUENCES\_TFRECORD \
    --recursive
\end{lstlisting}
\newpage
次に作成したNoteSequenceのデータセットを学習用と評価用に分割するために以下に示すのコマンドを実行する.\\
 --configで使用するRNNを指定する.--input\_dirでNoteSequenceの絶対パスを指定し,--output\_fileで分割したNotesequenceの出力先のディレクトリを指定する.
--eval\_ratioでNotesequenceのデータを何パーセント学習用に用いるかを指定する.コマンドの場合は10\%が学習用のデータになる.
\begin{lstlisting}[basicstyle=\ttfamily\footnotesize,frame=single]
melody_rnn_create_dataset \
--config=<one of 'basic_rnn', 'lookback_rnn', or 'attention_rnn'> \
--input=/tmp/notesequences.tfrecord \
--output_dir=/tmp/melody_rnn/sequence_examples \
--eval_ratio=0.10
\end{lstlisting}
\subsection{basic\_rnnを用いて学習を開始}
作成したNoteSequenceから学習モデルを作成するために以下のコマンドを実行する.\\
--configで学習に使用するbasic\_rnnを指定,--rundirで学習のために用意したNotesequenceを指定し,--sequence\_examplefileで学習モデルの出力先のディレクトリを指定する.
--hparamsでメモリの使用量を指定し,--rnn\_layer\_sizeで中間層のノード数を指定し,--num\_trainingstepsで学習回数を設定する.\\
\begin{lstlisting}[basicstyle=\ttfamily\footnotesize,frame=single]
melody_rnn_train \
--config=basic_rnn \
--run_dir=/tmp/melody_rnn/logdir/run1 \
--sequence_example_file=/tmp/melody_rnn/training_melodies.tfrecord \
--hparams="batch_size=64,rnn_layer_sizes=[64,64]" \
--num_training_steps=20000
\end{lstlisting}
\newpage
\subsection{lookback\_rnnを用いて学習を開始}
作成したNoteSequenceから学習モデルを作成するために以下のコマンドを実行する.\\
 --configで学習に使用するlookback\_rnnを指定,--rundirで学習のために用意したNotesequenceを指定し,--sequence\_examplefileで学習モデルの出力先のディレクトリを指定する.
--hparamsでメモリの使用量を指定し,--rnn\_layer\_sizeで中間層のノード数を指定し,--num\_trainingstepsで学習回数を設定する.\\
\begin{lstlisting}[basicstyle=\ttfamily\footnotesize,frame=single]
melody_rnn_train \
--config=lookback_rnn \
--run_dir=/tmp/melody_rnn/logdir/run1 \
--sequence_example_file=/tmp/melody_rnn/training_melodies.tfrecord \
--hparams="batch_size=64,rnn_layer_sizes=[64,64]" \
--num_training_steps=20000
\end{lstlisting}
\subsection{attention\_rnnを用いて学習を開始}
作成したNoteSequenceから学習モデルを作成するために以下のコマンドを実行する.\\
 --configで学習に使用するattention\_rnnを指定,--rundirで学習のために用意したNotesequenceを指定し,--sequence\_examplefileで学習モデルの出力先のディレクトリを指定する.
--hparamsでメモリの使用量を指定し,--rnn\_layer\_sizeで中間層のノード数を指定し,--num\_trainingstepsで学習回数を設定する.\\
\begin{lstlisting}[basicstyle=\ttfamily\footnotesize,frame=single]
melody_rnn_train \
--config=attention_rnn \
--run_dir=/tmp/melody_rnn/logdir/run1 \
--sequence_example_file=/tmp/melody_rnn/training_melodies.tfrecord \
--hparams="batch_size=64,rnn_layer_sizes=[64,64]" \
--num_training_steps=20000
\end{lstlisting}
\newpage
\subsection{音楽データの作成}
以下に示すコマンドで学習モデルに入力する.\\
 --configで学習に使用するattention\_rnnを指定,--rundirで学習済みのモデルを指定し,--output\_dirで音楽データの出力先のディレクトリを指定する.
--num\_outputsで生成する音楽データの個数を指定し,--num\_stepsで
--hparamsでメモリの使用量を指定し,--rnn\_layer\_sizeで中間層のノード数を指定し,--primer\_melodyで学習モデルに入力する最初の音程をMIDIの形式で指定する.\\
\begin{lstlisting}[basicstyle=\ttfamily\footnotesize,frame=single]
    melody_rnn_generate \
    --config=attention_rnn \
    --run_dir=/tmp/melody_rnn/logdir/run1 \
    --output_dir=/tmp/melody_rnn/generated \
    --num_outputs=10 \
    --num_steps=128 \
    --hparams="batch_size=64,rnn_layer_sizes=[64,64]" \
    --primer_melody="[60]"
\end{lstlisting}
\subsection{事前に学習済のモデルを使用}
また,Magendaのプロジェクトにすでに学習済のモデルが存在するのでそれを使用して音楽データを作成することもできる.
生成にはmagバンドファイルが必要になるのでmagendaのGithubに公開されているので,それをダウンロードしてくる.
その後以下に示すコマンドを実行する事で生成することができる.\\
 --configで学習に使用する学習モデルを指定,--rundirで学習済みのモデルを指定し,--output\_dirで音楽データの出力先のディレクトリを指定する.
--num\_outputsで生成する音楽データの個数を指定し,--num\_stepsで
--hparamsでメモリの使用量を指定し,--rnn\_layer\_sizeで中間層のノード数を指定し,--primer\_melodyで学習モデルに入力する最初の音程をMIDIの形式で指定する.\\
\begin{lstlisting}[basicstyle=\ttfamily\footnotesize,frame=single]
    melody_rnn_generate \
    --config=${CONFIG} \
    --bundle_file=${BUNDLE_PATH} \
    --output_dir=/tmp/melody_rnn/generated \
    --num_outputs=10 \
    --num_steps=128 \
    --primer_melody="[60]"
\end{lstlisting}
\newpage
\section{Polyphony\_rnnを使用して学習モデルを作成}
PolyphonyRNNを使用してする手順としては,大まかな流れは同じであるが--configで使用するRNNを指定する必要がないという違いがある
\subsection{NoteSequenceの作成}
NoteSequenceの作成は以下に示すコマンドで作成できる.\\
--input\_dirで学習させるMIDIデータのディレクトリの絶対パスを指定し,--output\_fileでNotesequenceの出力先のディレクトリを指定する.\\
\begin{lstlisting}[basicstyle=\ttfamily\footnotesize,frame=single]
    convert\_dir\_to\_note\_sequences \
    --input\_dir=\$INPUT\_DIRECTORY \
    --output\_file=\$SEQUENCES\_TFRECORD \
    --recursive
\end{lstlisting}
 次に作成したNoteSequenceのデータセットを学習用と評価用に分割するために以下に示すのコマンドを実行する.\\
--input\_dirでNoteSequenceの絶対パスを指定し,--output\_fileで分割したNotesequenceの出力先のディレクトリを指定する.
--eval\_ratioでNotesequenceのデータを何パーセント学習用に用いるかを指定する.コマンドの場合は10\%が学習用のデータになる.
\begin{lstlisting}[basicstyle=\ttfamily\footnotesize,frame=single]
    polyphony_rnn_create_dataset \
    --input=/tmp/notesequences.tfrecord \
    --output_dir=/tmp/polyphony_rnn/sequence_examples \
    --eval_ratio=0.10
    \end{lstlisting}
\subsection{学習の開始}
作成したNoteSequenceから学習モデルを作成するために以下のコマンドを実行する.\\
--configで学習に使用する学習モデルを指定,--rundirで学習のために用意したNotesequenceを指定し,--sequence\_examplefileで学習モデルの出力先のディレクトリを指定する.
--hparamsでメモリの使用量を指定し,--rnn\_layer\_sizeで中間層のノード数を指定し,--num\_trainingstepsで学習回数を設定する.\\
\begin{lstlisting}[basicstyle=\ttfamily\footnotesize,frame=single]
polyphony_rnn_train \
--run_dir=/tmp/polyphony_rnn/logdir/run1 \
--sequence_example_file=/tmp/polyphony_rnn/training_tracks.tfrecord \
--hparams="batch_size=64,rnn_layer_sizes=[64,64]" \
--num_training_steps=20000
\end{lstlisting}
\subsection{音楽データの作成}
以下に示すコマンドで学習モデルに入力する.\\
--configで学習に使用する学習モデルを指定,--rundirで学習済みのモデルを指定し,--output\_dirで音楽データの出力先のディレクトリを指定する.
--num\_outputsで生成する音楽データの個数を指定し,--num\_stepsで
--hparamsでメモリの使用量を指定し,--rnn\_layer\_sizeで中間層のノード数を指定し,--primer\_melodyで学習モデルに入力する最初の音程をMIDIの形式で指定する.(図\ref{fig:MIDIと音階}参照)\\
\begin{lstlisting}[basicstyle=\ttfamily\footnotesize,frame=single]
    polyphony_rnn_generate \
    --run_dir=/tmp/polyphony_rnn/logdir/run1 \
    --hparams="batch_size=64,rnn_layer_sizes=[64,64]" \
    --output_dir=/tmp/polyphony_rnn/generated \
    --num_outputs=10 \
    --num_steps=128 \
    --primer_pitches="[67,64,60]" \
    --condition_on_primer=true \
    --inject_primer_during_generation=false
\end{lstlisting}
\chapter{結論}
\section{AIによる楽曲制作の結果}
\subsection{モデルによる生成結果の違い}
\subsection{学習回数による生成結果の違い}
\begin{figure}[h]
    \begin{screen}
    \begin{center}
        \includegraphics[scale=0.5, clip]{./img/basicMIDI.png}
        \caption{basic\_rnnによる学習回数ごとの生成結果}
        \label{fig:basic_rnnによる学習回数ごとの生成結果}
    \end{center}
    \end{screen}
\end{figure}
\begin{figure}[h]
    \begin{screen}
    \begin{center}
        \includegraphics[scale=0.5, clip]{./img/lookbackMIDI.png}
        \caption{lookback\_rnnによる学習回数ごとの生成結果}
        \label{fig:lookback_rnnによる学習回数ごとの生成結果}
    \end{center}
    \end{screen}
\end{figure}
\begin{figure}[h]
    \begin{screen}
    \begin{center}
        \includegraphics[scale=0.5, clip]{./img/attentionMIDI.png}
        \caption{attention\_rnnによる学習回数ごとの生成結果}
        \label{fig:attention_rnnによる学習回数ごとの生成結果}
    \end{center}
    \end{screen}
\end{figure}
  
\subsection{ノード数による生成結果の違い}
\section{調査結果}

\newpage

\section{今後の課題}
aaa\\

\newpage

%ここまで内容


%%%%%%%%%%%%%%%%%%%%%%%%%%%%%%%%%%%%%%%%%%%%%%%%%%%%%%%%%%%%%%%%%%%%%%%

\chapter*{ \\謝辞}\addcontentsline{toc}{chapter}{謝辞}

%%%%%%%%%%%%%%%%%%%%%%%%%%%%%%%%%%%%%%%%%%%%%%%%%%%%%%%%%%%%%%%%%%%%%%%

\begin{thebibliography}{99}\addcontentsline{toc}{chapter}{参考文献}%参考文献

%\bibitem{tankoubon} %キーワード:文章中で呼ぶ時に使う
%(単行本の場合)著者名(発行年)書名,発行所

%\bibitem{ronbun}
%(論文の場合)著者(発表年)タイトル,雑誌名,巻数,論文所在ページ

%\bibitem{webpage}
%(webページの場合)著者(発行年)表題.\\
%(改行して)http://・・・
\bibitem{ronbun1}
A. M. Turing (1950)"COMPUTING MACHINERY AND INTELLIGENCE",\\
https://www.csee.umbc.edu/courses/471/papers/turing.pdf
\bibitem{ronbun2}
J. McCarthy,Dartmouth College M. L. Minsky,Harvard University N.\\
Rochester,I.B.M. Corporation C.E. Shannon,Bell Telephone Laboratories(1955),
"A PROPOSAL FOR THE DARTMOUTH SUMMER RESEARCH PROJECT ON ARTIFICIAL INTELLIGENCE"\\
http://www-formal.stanford.edu/jmc/history/dartmouth/dartmouth.html
\bibitem{ronbun3}
Quoc Le Marc'Aurelio Ranzato Rajat Monga Matthieu Devin Kai Chen Greg Corrado Jeff Dean Andrew Ng(2012)
"Building High-level Features Using Large Scale Unsupervised Learning"\\
https://storage.googleapis.com/pub-tools-public-publication-data/pdf/38115.pdf
\bibitem{webpage1}
Google Brainチーム"Magenda",\\
https://github.com/tensorflow/magenta
\bibitem{webpage2}
財経新聞"人工知能を使って執筆した小説が星新一賞一次選考を通過",\\
https://www.zaikei.co.jp/article/20160323/299468.html
\bibitem{webpage3}
YAHOOニュース"AI絵画、大手オークションで初の落札 予想額の40倍超",\\
https://headlines.yahoo.co.jp/hl?a=20181026-00010002-afpbbnewsv-int
\bibitem{webpage4}
(社) 人工知能学会”人工知能の話題”,\\
https://www.ai-gakkai.or.jp/whatsai/AItopics5.html
\bibitem{webpage5}
マイナビニュース"XMLはもう不要!? Google製シリアライズツール「Protocol Buffer」",\\
https://news.mynavi.jp/article/20080718-protocolbuffer/
\bibitem{webpage6}
日本経済新聞(2011)"人間にクイズで勝ったコンピューター「ワトソン」の素顔",\\
https://www.nikkei.com/article/DGXNASDD2305K\_T20C11A3000000/
%\bibitem{VM}
%大久保 健一,大塚 弘毅,染谷 文昭,照川 陽太郎“できるPROシリーズVMware vSphere6”.
\end{thebibliography}

%%%%%%%%%%%%%%%%%%%%%%%%%%%%%%%%%%%%%%%%%%%%%%%%%%%%%%%%%%%%%%%%%%%%%%%

\end{document}