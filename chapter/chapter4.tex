\chapter{楽曲制作}
\section{NoteSequenceについて}
NoteSequenceとはMIDIデータから作成されるプロトコルバッファである.
プロトコルバッファとはGoogleが2008年にオープンソース化したバイナリベースのデータフォーマットである.
既存の技術としてはXMLやJSONなどのテキストベースのデータフォーマットがあるが,プロトコルバッファはバイナリフォーマットであるので,アプリケーション間でデータ構造の送受信をする際に少ないデータ量ですむという特徴がある。
\section{NoteSequenceの作成}
NoteSequenceの作成は図のコマンドで作成できる.
\begin{figure}[!ht]
    \begin{screen}
    \begin{center}
        \includegraphics[scale=0.7, clip]{./img/Notesequence_make.png}
        \caption{NoteSequenceの作成}
        \label{fig:NoteSequenceの作成}
    \end{center}
    \end{screen}
\end{figure}\\
--input\_dirで学習させるMIDIデータの絶対パスを指定し,--output\_fileでNotesequenceの出力先のディレクトリを指定する.\\
\newpage
次に作成したNoteSequenceのデータセットを学習用と評価用に分割するために,下記のコマンドを実行する.
\begin{figure}[!ht]
    \begin{screen}
    \begin{center}
        \includegraphics[scale=0.7, clip]{./img/Notesequence_split.png}
        \caption{NoteSequenceを学習用と評価表に分割}
        \label{fig:NoteSequenceを学習用と評価表に分割}
    \end{center}
    \end{screen}
\end{figure}\\
--configで使用するRNNを指定する.--eval\_ratioでNotesequenceの10%が評価のためのデータになり,残りの90%が学習用のデータになる.\\
\subsection{学習の開始}
作成したNoteSequenceから学習モデルを作成するために下記のコマンドを実行する.
\begin{figure}[!ht]
    \begin{screen}
    \begin{center}
        \includegraphics[scale=0.5, clip]{./img/Rnn_train.png}
        \caption{BasicRNNを使用した学習の開始}
        \label{fig:BasicRNNを使用した学習の開始}
    \end{center}
    \end{screen}
\end{figure}
--configで学習に使用する学習モデルを指定,--rundirで,--sequence\_examplefileで学習のために用意したNotesequenceを指定する.
--hparamsでメモリの使用量を指定し,--rnn\_layer\_sizeで中間層のノード数を指定し,--num\_trainingstepsで学習回数を設定する.\\
\newpage
\subsection{音楽データの作成}
下図のコマンドで学習モデルに入力する 音調をMIDIの形式で 指定し音楽データを作成する.
\begin{figure}[!ht]
    \begin{screen}
    \begin{center}
        \includegraphics[scale=0.7, clip]{./img/MIDI_make.png}
        \caption{学習モデルを使用し,10曲を作成}
        \label{fig:学習モデルを使用し,10曲を作成}
    \end{center}
    \end{screen}
\end{figure}